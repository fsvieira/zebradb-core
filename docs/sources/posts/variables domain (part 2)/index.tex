\documentclass{article}
\usepackage[utf8]{inputenc}
\usepackage{tikz}
\usetikzlibrary{automata,positioning}
\begin{document}

On the last post I talked about how to extract and construct a variable domain, and I was very happy with 
the end result. 

Now Zebradb-core needs to handle the domains on unification operation/phase.
However the conclusion I got was not completely what I would expect and now I need to make some changes. 

\section{Introduction}

The Zebradb-core solves queries by unifying them with definitions. 
The unification operation is defined as a binary operation that can be applied to variables, tuples and constants. 
Now with the introduction of domains I need to extend the unify operation to handle the new type.

\section{Domain Unification: The problem}

As described on last post domains are a set of possible values that a variable can have, however the way 
domains are constructed they only contain tuples or constants, and they are never empty. 

For simplicity I will use the symbol $\otimes$ to represent unify operation, 
$p$ as variable, $t$ as tuple, $c$ as constant and $d$ as domain.

\begin{enumerate}
\item variables unify with anything:
    \begin{enumerate}
        \item $d \otimes p = d$, $p=d$,
        \item $p \otimes d = d$, $p=d$.
    \end{enumerate}

\item constants, if constant is in domain then constant is the result:
    \begin{enumerate}
        \item $c \otimes d = c$ iff $c \in d$,
        \item $d \otimes c = c$ iff $c \in d$.
        \item else unify fails.
    \end{enumerate}

\item tuples: 
    \begin{enumerate}
        \item $t \otimes d = \{x: \forall p \in d, p \otimes t \}$,
        \item $d \otimes t = \{x: \forall p \in d, p \otimes t \}$,
        \item only successful unification are part of resulting domain,
        \item if resulting domain is empty then fail.
    \end{enumerate}

\item domains:
    \begin{enumerate}
        \item $d_{0} \otimes d_{1} = \{x: \forall p \in d_{0}, \forall q \in d_{1}, p \otimes q \}$,
        \item only successful unification are part of resulting domain,
        \item if resulting domain is empty then fail.
    \end{enumerate}

\end{enumerate}

The tricky part is on tuple unification with domains, the problem is that Zebradb-core needs to guarantee 
the consistency of domains across the all tuple, from top to bottom.

Basically a domain can contain tuples with bound and free variables, on the case of unification of a domain 
with only one tuple free variables are not a problem, however bound variables can became a problem, a bound 
variable can unify with multiple terms making the tuple and domain inconsistent, this problem would need 
to be solved with the reconstruction of the domain involving the creation of new branches on unification phase
and recalculating the domain, which would defeat the purpose of using domains.

Basically a domain can contain tuples with bound and free variables, on the case of free variables we need 
to guarantee that they are not repeated on other domain tuples, this can be solved by replacing tuple 
variables with new ones, before unification. The case of bound variables is more complicated, to guarantee 
tuple consistency we need to recalculate the domain involving the creation of new branches on unification phase
which would defeat the purpose of using domains.

\section{Domain Unification: The Solution}

The solution is very simple, instead of trying to solve tuple unification with domains that brings a lot 
of complicated issues and not sure if any benefit, I just ditched tuples from domains.

So now variable domains can only contain constants. 
Operation over domains like intersection and union is much more simple. 
Also I believe constants are the most influencing aspect of branches growth.

So unification now becomes something like this:

\begin{enumerate}
\item variables unify with anything:
    \begin{enumerate}
        \item $d \otimes p = d$, $p=d$,
        \item $p \otimes d = d$, $p=d$.
    \end{enumerate}

\item constants, if constant is in domain then constant is the result:
    \begin{enumerate}
        \item $c \otimes d = c$, iff $c \in d$,
        \item $d \otimes c = c$, iff $c \in d$.
        \item else unify fails.
    \end{enumerate}

\item tuples: 
    \begin{enumerate}
        \item Fail to all tuples.
    \end{enumerate}

\item domains:
    \begin{enumerate}
        \item $d_{0} \otimes d_{1} = d_{0} \cup d_{1}$ , iff $d_{0} \cup d_{1} \neq \emptyset$
        \item else unify fails.
    \end{enumerate}

\end{enumerate}

So lets see some examples:

\begin{enumerate}
\item $\{1, 2, 4\} \otimes \{1, 4, 5\} = \{1, 4\}$,
\item $\{1, 2, 4\} \otimes \{8, 7\}$, fail because intersection is empty,
\item $\{1, 2, 4\} \otimes v = \{1, 2, 4\}$,
\item $\{1, 2\} \otimes (v_{0} \ldots v_{n})$, fail, tuples are not unifiable with domains,
\item $\{1, 2\} \otimes 1 = 1$,
\item $\{1, 2\} \otimes 4$, fail because $4 \not\in \{1, 2\}$

Now I just need to implement this changes and handle unification, since this is such simple operations I hope 
that this will improve greatly the performance Zebradb-core.

Thats it.

\end{enumerate}



\end{document}