\documentclass{article}
\usepackage[utf8]{inputenc}
\usepackage{tikz}
\usetikzlibrary{automata,positioning}
\begin{document}

\section{Negations}


On my last posts I talked about domains, and how to construct them, but I didn't mention negations, that is a big part of domains.

Negation has always been the part of Zebradb-core that I feel its a little bit clunky, but I didn't found any other elegant solution
and even if is not perfect I feel that is sound.

There is two problems with negations, first on domain construction and second on negation evaluation.

\section{Domain construction and negations}

The construction of domains is a phase where domains are extracted from multiple branches and "compressed" into a single branch, 
to keep things consistent the created branches must maintain the original meaning from the source branches, so 
we also need to take negation into account on this phase.

The solution is simple if a group of branches generates same domains but different negations then the group is splitted and will 
generate different branches with same domains and different negation set.

\section{Negation evaluation}

Now that we have domains, we need to deal with them on negation evaluation for example:

(color yellow)
(color red)
(color blue)

('x = 'x)
('x != 'y ^('x = 'y))

?((color 'x) != (color 'y))

=> ((color [yellow, red, blue]) != (color [yellow, red, blue]) ^((color [yellow, red, blue]) = (color [yellow, red, blue])))
=> [yellow, red, blue] = [yellow, red, blue] => [yellow, red, blue]

In this example the negation unify two domains that are equals and therefor the negated tuple will succeed making negation to fail.
But this is not correct, while tuple should succeed negation shouldn't fail, because we can find many failing values on this domain combinations,
for example: 

=> [yellow, red, blue] = [yellow, red, blue] => [yellow, red, blue]
    fail -> yellow = [red, blue]
    fail -> red = [yellow, blue]
    ...

This kind of domain construction, where we invert the combination of successful domains to fail I call it flop.

The flop is calculated for each successful branch of a negation evaluation, and all resulting flops are then merged/multiplied 
with each other, so that a common failing state is found, if a flop is not found than negation will fail.

All negations flops are also merged/multiplied by each, so if a tuple has more then one negation and they all have flop values than 
their merge/multiplication must also succeed.

The idea of multiplication is very simple, for a negation to succeed the negated tuple can't exist, meaning it can't 
have any solution, the flop is a set of a combination of failing domains, and flops are multiplied with each other 
for example:

flopA = [A, ... A_n], flopB = [B ... B_n]

flopA * flopB = [A * B, ... , A * B_n, A_1 * B, ... A_1 * B_n, ... A_n * B_n]

A = \{
    domainA: [0, 1, 2],
    domainB: [3, 4],
    domainC: [6]
\}

B = \{
    domainA: [0, 2],
    domainB: [3, 5]
\}

A * B = \{
    domainA: [0, 2],
    domainB: [3],
    domainC: [6]
\}

So equal domains are intersected, if a domain is not set on A or B then the result is the domain itself.
If the intersection of two domains is empty then this is not a valid domain combination, in the end if flop is empty 
then negation fails.


\section{Conclusion}

With this solutions we are able to correctly handle domains with and on negations, the cool part of negation evaluation 
is that we can check and negate a bunch of values with less branches, and since we are using only constants the checks should be 
pretty fast.

\end{document}