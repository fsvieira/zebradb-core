\documentclass[11pt,a4paper]{report}
\usepackage{hyperref}
\usepackage{amsmath}

\newcommand{\zsystem}{Zebra-machine}

\begin{document}

\title{The Great Book of Zebra}
\author{The Zebra Project}
\date{\today}
\maketitle

\chapter*{Preface}

This book is a collaborative work from the \href{zebrajs}{https://github.com/fsvieira/zebrajs} project community 
and everyone is invited to participate.

The list of contributors is at the contributors section~\ref{sec:contributors} and your name can be there too :D.

This is a work in progress.

\chapter{Introduction}

\section{The \zsystem}

This is the official book of \zsystem. Here you will find anything you need to understand in deep the \zsystem , 
the book covers both theoretical and practical definitions. 

\section{The \zsystem}

The \zsystem consists of a language of \zsystem\ terms, which is defined by a certain formal syntax, and a set of transformation rules,

\begin{enumerate}
\item $'p$, $p$ its a varibable and its a \zsystem\ term.
\item $(p_{0} \ldots p_{n})$ its a \zsystem\ term and $p_{n}$ is a \zsystem\ term.
\item Anything else is a constant.
\end{enumerate}


\section{Contributors}
\label{sec:contributors}

\begin{itemize}
    \item \href{https://github.com/fsvieira}{Filipe Vieira, https://github.com/fsvieira}
\end{itemize}

\end{document}